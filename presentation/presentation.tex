\documentclass[10pt]{beamer}

\usepackage{beamerthemetree}
\usepackage{amsmath}
\usepackage{tikz}
\usepackage{graphics}
\usepackage{subfigure}
\usepackage{graphicx}
%\usepackage{rotating}
\usepackage{relsize}
%\usepackage{bbm}
\usepackage{ragged2e}
\usepackage{latexsym}
\usepackage{amssymb}
\usepackage[latin1]{inputenc}% if the file is really encoded this way

%\usepackage{bookmark}
%\usepackage{wrapfig}
\usepackage{booktabs}
\usepackage{tabularx}
\usepackage{verbatim}
\newcolumntype{Y}{>{\centering\arraybackslash}X}
%\newcommand{\lenitem}[2][.7\linewidth]{\parbox[t]{#1}{\strut #2\strut}}

\setbeamertemplate{footline}[page number]


\title[Combinatorial Bandits]{Combinatorial Bandit Algorithms in Practice}

\author[Angus McKay, Akhil Lohia]{
			Angus McKay \and Akhil Lohia}
%\institute{University of Cambridge}
\date{June 2017}

\usetheme{CambridgeUS}

\let\Tiny=\tiny
\begin{document}

\begin{frame}
  \titlepage
\end{frame}



\section{Introduction}


\begin{frame}{Introduction}
	\begin{itemize}
		\item Bandit Algorithms and Online Learning
		\begin{itemize}
			\item These problems have generated a lot of interest recently
		\end{itemize}
		\item Different variation - semi-bandit and full-bandit
	\end{itemize}
\end{frame}




\begin{frame}{Introduction}
	\begin{itemize}
		\item Stochastic combinatorial semi-bandit problems
		\begin{enumerate}
			\item Online learning problem where an agent chooses a subset of ground items at each step - subject to combinatorial constraints
			\item The stochastic weights of these items are observed and their sum is received as a payoff
		\end{enumerate}
	\end{itemize}
\end{frame}


\begin{comment}

\begin{frame}{Introduction}
	\begin{itemize}
		\item We develop a model which links wealth, marriage, and sex selection
		\begin{itemize}
			\item Dowries and sex selection are jointly determined in the model
			\item The root cause of sex selection is specific frictions in the marriage market
		\end{itemize}
		\item The model generates the prediction that sex selection is increasing with (relative) wealth
		\begin{itemize}
			\item The prediction is tested with unique data we have collected, covering the entire population of 1.1 million individuals residing in half a rural district in South India
		\end{itemize}
	\end{itemize}
\end{frame}




\begin{frame}{Introduction}
	\begin{itemize}
		\item The main empirical finding is that the probability that a child is a girl is decreasing as we move up the wealth distribution within castes, which define independent marriage markets in India
		\begin{itemize}
			\item The variation in sex ratios within castes in a single (unexceptional) district is comparable to the variation across all districts in the country
			\item Estimation of the model's structural parameters allows us to quantify the impact of alternative policies
		\end{itemize}
	\end{itemize}
\end{frame}






\section{A Theory of Wealth, Marriage, and Sex Selection}

\begin{frame}{A Theory of Wealth, Marriage, and Sex Selection}
	% A Theory of Wealth, Marriage, and Sex Selection
\end{frame}




\begin{frame}{Population}
	\begin{itemize}
		\item Each family consists of one parent and one child
		\item Denote the boy's family wealth by $x$ and the girl's by $y$
		\item The measure of families with boys and girls will be endogenous as will be the distribution of their wealth, $F(x)$ and $G(y)$, respectively
		\begin{itemize}
			\item Without sex selection, $F(\cdot) = G(\cdot)$
		\end{itemize}
	\end{itemize}
\end{frame}




\begin{frame}{Preferences, Payoffs, and Consumption}
	\begin{itemize}
		\item Denote the wealth-contingent consumption of parents by $C_x$, $C_y$ and that of the children by $c_x$, $c_y$
		\item All individuals have logarithmic preferences over consumption and parents, in addition, have altruistic preferences over the consumption of their children
		\begin{align*}
			U = \log(C_i) + \log(c_i) ,  \hspace{1cm}  \forall i=\left\lbrace x , y \right\rbrace
		\end{align*}
	\end{itemize}
\end{frame}




\begin{frame}{The Marriage Institution}
	\begin{itemize}
		\item The model incorporates the key features of the marriage institution in India
		\begin{itemize}
			\item Castes form independent marriage markets and the model describes one such market
			\item Marriages are arranged, with family wealth being a major consideration when forming a match
			\item Marriage is patrilocal; i.e. women move into their husband's homes
		\end{itemize}
	\end{itemize}
\end{frame}




\begin{frame}{The Marriage Institution}
	\begin{itemize}
		\item The benefit of patrilocal marriage is that if a girl matches with a wealthy boy she will get to consume a fraction of the wealth her husband receives as a transfer from his parents
		\begin{itemize}
			\item denote the transfer by t
			\item The boy obtains a fraction $\alpha \geq \frac{1}{2}$ of the transfer, while the wife cannot be prevented from consuming a fraction $1-\alpha$ of what her husband receives
		\end{itemize}
		\item The cost of patrilocal marriage is that the boy's parent is only willing to accept the match if the girl's parent pays a dowry $d$
	\end{itemize}
\end{frame}






\begin{frame}{Consumption Levels}
	\begin{itemize}
		\item The consumption of all agents of a married groom-bride pair $(x,y)$ can be written as
		\begin{align*}
			c_x &= \alpha t 	\\
			C_x &= x - t + d 	\\
			c_y &= (1-\alpha)t 	\\
			C_y &= y - d 		\\
		\end{align*}
		\item We solve backwards, starting with the choice of transfer, $t$, and then deriving the dowry, $d$, in competitive equilibrium
	\end{itemize}
\end{frame}








\begin{frame}{Matching and Sex Selection}
	\textbf{Proposition 1.} \emph{In equilibrium, there is sex selection at every wealth level and Positive Assortative Matching on wealth, which implies hypergamy}
	\vspace{0.4cm}
	\begin{itemize}
		\item Sex selection is obtained because of the social norm that all girls must marry
		\item PAM is obtained because wealthy parents are willing to pay a higher dowry to match with wealthy families to ensure higher consumption for their daughters
		\item The dowry must be increasing steeply enough in $x$ to ensure that the matching is stable
	\end{itemize}
\end{frame}




\begin{frame}{Wealth and Sex Selection}
	\begin{itemize}
		\item How does sex selection vary across the wealth distribution?
		\begin{itemize}
			\item Suppose that sex selection is the same at all wealth levels
			\item The wealthiest boys match with the wealthiest girls, but because of the shortage of girls, all other boys marry down
			\item Given assortative matching, the wealth-gap increases as we move down the distribution, making it more attractive for poorer parents to have a girl
			\item Thus, sex selection will decline endogenously as we move down the distribution
		\end{itemize}
	\end{itemize}
\end{frame}







\section{Results}



\begin{frame}{Matching Patterns}
	\begin{figure}
		% \caption{}
		\includegraphics[height=8cm,trim={2cm 7cm 2cm 7cm}, clip]{Figures/Match}
	\end{figure}
\end{frame}



\begin{frame}{Dowries and Sex Ratios}
	\begin{figure}
		% \caption{}
		\includegraphics[height=8cm,trim={2cm 7cm 2cm 7cm}, clip]{Figures/dstat+Pr}
	\end{figure}
\end{frame}




\begin{frame}{Dowries and Sex Ratios - varying  $\alpha$}
	\begin{figure}
		% \caption{Parameter: alpha}
		\includegraphics[height=8cm,trim={2cm 7cm 2cm 7cm}, clip]{Figures/d+Pr_alpha}
	\end{figure}
\end{frame}





\begin{frame}{Wealth and Sex Selection: Empirical Results}
	\begin{itemize}
		\item The study area is representative of rural Tamil Nadu and rural South India with respect to demographic and socioeconomic characteristics
		\item No castes that are strongly associated with sex selection are present in the area and sex ratios in Vellore district are around the median in Tamil Nadu
		\item The relationship between wealth and sex selection \textit{within} castes that we uncover is likely to be valid elsewhere
	\end{itemize}
\end{frame}





\begin{frame}{Wealth and Sex Ratios: Estimation}
	\begin{itemize}
		\item The benchmark equation that we estimate has the following specification:
		\begin{align*}
			Pr \left( G_{ij}=1 \right) = \beta R_{ij} + \delta_j + \epsilon_{ij}
		\end{align*}
		\begin{itemize}
			\item $ G_{ij}=1 $ if child $i$ from caste $j$ is a girl, 0 if a boy
			\item $ R_{ij} $ is the child's family's rank in its caste's wealth distribution
			\item $ \delta_j $ is a full set of caste dummies
		\end{itemize}
		\vspace{0.5cm}
		\item The sample is restricted to "single family" households, which account for 96.2\% of households with children, and families are ranked in the caste wealth distribution on the basis of their per capita wealth
	\end{itemize}
\end{frame}




\begin{frame}{Wealth and Sex Ratios: Augmented Specification}
	\begin{itemize}
		\item Factors that contributed to the increase in sex selection over time could also generate cross-sectional variation:
		\begin{enumerate}
			\item Reduced fertility coupled with the need for at least one son
			\item Improved access to sex selection technology
			\item Relative increase in the economic returns to boys versus girls
		\end{enumerate}
		\item Additional regressors:
		\begin{enumerate}
			\item Household wealth
			\item Parental education
			\item Child's birth-order
		\end{enumerate}
	\end{itemize}
\end{frame}





\begin{frame}{Wealth and Sex Ratios (Reported Wealth)}
	\begin{table}
		\begin{scriptsize}
		% \caption{Table5_bmark_reg}
		\input{Tables/Table5_bmark_reg_1}
		\end{scriptsize}
	\end{table}
\end{frame}



\begin{frame}{Wealth and Sex Ratios: Predicted Wealth}
	\begin{itemize}
		\item Household wealth is measured as the annual income from land and labour in the past year
		\item This will measure permanent, i.e. mean income, with error
		\item To purge the measurement error, we estimate the relationship between current household income and village revenue in 1871 interacted with caste fixed effects
		\begin{itemize}
			\item The F-statistic for the test of joint significance of historical wealth and the caste dummy interactions is $20.4$
		\end{itemize}
	\end{itemize}
\end{frame}




\begin{frame}{Wealth and Sex Ratios (Predicted Wealth)}
	\begin{table}
		\begin{scriptsize}
		% \caption{Table5_bmark_reg}
		\input{Tables/Table5_bmark_reg_2}
		\end{scriptsize}
	\end{table}
\end{frame}





\begin{frame}{Wealth and Sex Ratios: Robustness Tests}
	\begin{itemize}
		\item Family size will not be complete for some children aged 0-6
		\begin{itemize}
			\item We thus check the robustness of the results for 7-13 year olds
		\end{itemize}
		\item The demand for at least one son is another determinant of sex selection
		\begin{itemize}
			\item This mechanism does not affect first borns
		\end{itemize}
	\end{itemize}
\end{frame}




\begin{frame}{Wealth and Sex Ratios (Robustness Tests)}
	\begin{table}
		\begin{scriptsize}
		% \caption{Table6_rob_firstborns_7to13}
		\input{Tables/Table6_rob_firstborns_7to13_short}
		\end{scriptsize}
	\end{table}
\end{frame}




\begin{frame}{Wealth and Sex Ratios (All Castes)}
	\begin{figure}
		%\caption{Figure7b_sexselection_pooled}
		\includegraphics[height=6.5cm, trim={0cm 0 5cm 0}, clip]{"Figures/Figure7b_sexselection_pooled"}
		% CHANGE REQUEST: change blue line to dashed, as in dowry figure
	\end{figure}
\end{frame}




\begin{frame}{Wealth and Sex Ratios (Aged 0-6)}
	\begin{figure}
		%\caption{Figure7a_sexselection_indiv}
		\includegraphics[height=6.5cm,keepaspectratio]{"Figures/Figure7a_sexselection_indiv"}
	\end{figure}
\end{frame}




\begin{frame}{Wealth and Sex Ratios (Aged 7-13)}
	\begin{table}[X]
		%\caption{Figure7a_sexselection_indiv}
		\includegraphics[height=6.5cm,keepaspectratio]{"Figures/FigureA1_sexselection_7to13"}
	\end{table}
\end{frame}




\begin{frame}{Magnitude of Within-Caste Variation}
	\begin{itemize}
		\item To quantify the magnitude of the within-caste variation, we partition each caste into eight equally sized wealth classes
		\item Compare $R^2$ with and without caste fixed effects
		\begin{itemize}
			\item Within-caste variation accounts for 70\% of explained variation with 30 largest castes
			\item 87\% with 12 largest castes
		\end{itemize}
		\item Measure the range of sex ratios across the wealth classes
		\begin{itemize}
			\item 97 to 117
		\end{itemize}
	\end{itemize}
\end{frame}




\begin{frame}{Structural Estimates}
	\begin{table}
	\scalebox{.52}{
		% \caption{Parameter: a}
		%\begin{footnotesize}
		\input{Tables/pred_sex_ratios_2_both_TEX}
		%\end{footnotesize}
	}
	\end{table}
\end{frame}





\begin{frame}{Counter-Factual Policy Experiments}
	\begin{enumerate}
		\item Gift tax on the dowry
		\item Conditional Cash Transfer schemes
		\begin{itemize}
			\item Parents receive transfers at different points in childhood, conditional on having a girl
			\item An insurance cover is provided, which matures when the girl turns 18 or 20
			\item Some schemes are restricted to low income families
		\end{itemize}
	\end{enumerate}
\end{frame}





\begin{frame}{Counter-Factual (Tax)}
	\begin{figure}
		% \caption{Parameter: alpha}
		\includegraphics[height=8cm,trim={2cm 7cm 2cm 7cm}, clip]{Figures/counter_tax}
	\end{figure}
\end{frame}





\begin{frame}{Counter-Factual (Transfers)}
	\begin{figure}
		% \caption{Parameter: alpha}
		\includegraphics[height=8cm,trim={2cm 7cm 2cm 7cm}, clip]{Figures/counter1}
	\end{figure}
\end{frame}


\end{comment}

\begin{frame}{Conclusion}
	\begin{itemize}
		\item Different algorithms perform better in different scenarios
		\item Adverserial and stochastic settings must be handled differently
		\begin{itemize}
			\item We compare CombUCB1, FPL-Trix and CombLinTS
		\end{itemize}
		\item next point
		\begin{itemize}
			\item sub-point
		\end{itemize}
	\end{itemize}
\end{frame}




\begin{frame}{Conclusion}
	\begin{itemize}
		\item Final remarks
		\begin{enumerate}
			\item Remark 1
			\item Remark 2
		\end{enumerate}
	\end{itemize}
\end{frame}





\end{document}
